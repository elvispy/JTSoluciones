\section{Residuos Cuadraticos y el simbolo de Legendre}

\begin{sol}
	\begin{itemize}
	\item $\left(\dfrac{44}{103}\right)= \left(\dfrac{11}{103}\right) \cdot \left(\dfrac{4}{103}\right) = \left(\dfrac{11}{103}\right) = \frac{(-1)^{\frac{11-1}{2} \cdot \frac{103-1}{2}}}{\left(\dfrac{103}{11}\right) } = \frac{-1}{\left(\dfrac{4}{11}\right)} = - \left(\dfrac{4}{11}\right) = -1$
	\item $\left(\dfrac{-60}{1019}\right) = \left(\dfrac{-90}{1019}\right) \cdot \left(\dfrac{2}{1019}\right) \cdot \left(\dfrac{3}{1019}\right) = 1 \cdot -1 \cdot -\left(\dfrac{1019}{3}\right) = -1.$
	
	Recordando que $\left(\dfrac{a}{p}\right) = \left(\dfrac{a}{p}\right)^{-1}$ y $\left(\dfrac{-120}{1019}\right) = \left(\dfrac{-40}{1019}\right) \cdot \left(\dfrac{2}{1019}\right)$
	
	\item $\left(\dfrac{2010}{1019}\right) = \left(\dfrac{44}{103}\right) = \left(\dfrac{-28}{1019}\right) = \left(\dfrac{-1}{1019}\right) \cdot \left(\dfrac{7}{1019}\right) = - \left(\dfrac{7}{1019}\right) = \left(\dfrac{1019}{7}\right) = \left(\dfrac{4}{7}\right) = 1$
	\end{itemize}
\end{sol}


\begin{sol}
	Notemos que $\varphi(49)  = 42$. Ademas, si $x$ satisface la ecuaci\'on del problema, entonces $(x, 7) = 1$. Eso implica que podemos afirmar que el pequeño teorema de Fermat es satisfecho. Es decir, $x^{42} \equiv 1 \implies x^2 \equiv 1 \implies (x+1)(x-1) \equiv 0  \implies x \equiv \pm 1$, debido a que $(x+1, x-1) \leq 2$. Verificamos que esos dos casos son en realidad los \'unicos posibles.
\end{sol}

\begin{sol}
	Observe que $\left(\dfrac{\frac{p-1}{4}}{p}\right) = \left(\dfrac{\frac{p-1}{4}}{p}\right) \cdot \left(\dfrac{4}{p}\right) = \left(\dfrac{-1}{p}\right) = (-1)^{2n} = 1$. 
	
	Entonces $\exists a, a^{2} = \left(\dfrac{\frac{p-1}{4}}{p}\right)$, por lo tanto $n^{n} = \left(\frac{p-1}{4}\right)^{\frac{p-1}{4}} = a^{2a^{2}} $
\end{sol}

\begin{sol}
	Note que $\left(\dfrac{236}{257}\right)  = \left(\dfrac{4}{257}\right) \cdot  \left(\dfrac{59}{257}\right) = \left(\dfrac{59}{257}\right) = \left(\dfrac{257}{59}\right) = \left(\dfrac{21}{59}\right) = \left(\dfrac{3}{59}\right) \cdot \left(\dfrac{7}{59}\right) = \left(\dfrac{59}{3}\right) \cdot \left(\dfrac{59}{7}\right) = \left(\dfrac{2}{3}\right) \cdot \left(\dfrac{3}{7}\right) = \left(\dfrac{-3}{7}\right) = \left(\dfrac{4}{7}\right) = 1$
\end{sol}

\begin{sol}
	Supongamos que tenemos una lista finita de primos de la forma $3k-1$, $\{p_{i}\}_{i\leq n}$. Definamos el numero $n = \left( \prod_{i=1}^{n} p_{i} \right)^{2} + 1$. Claramente $(n, p_{i}) = 1, \left(\dfrac{n}{3}\right) = -1$. Sigue que $\exists q | n$, primo tal que $ \left(\dfrac{q}{3}\right) = 1$ (si no existiese tendriamos $\left(\dfrac{n}{3}\right) = 1$). Pero como $q$ es coprimo con todos los numeros anteriores de la lista, tenemos que $q$ no pertenece a la lista y $q \equiv -1 (\mod 3)$, lo cual es una contradicci\'on.
	
	Ahora supongamos que la lista est\'a conformada \'unicamente por elementos de la forma $3k+1$. Defina $N = (2 \prod_{i}^{n} p_{i})^{2} + 3$. Tenemos que $N \equiv 1 (\mod 3)$. Supongamos que $q|N$, es decir, $-3 \equiv (2 \prod_{i}^{n} p_{i})^{2} (mod$ $q)$. Es decir, $\left(\dfrac{-3}{q}\right) = 1$. Observe que $\left(\dfrac{-3}{q}\right) = (-1)^{q-1} \cdot \left(\dfrac{q}{3}\right) = q (\mod 3) \implies q = 3k+1$. Finalmente, vea que $(q, p_{i}) \leq (N, p_{i}) = (3,p_{i}) = 1$. Es decir, $q$ es coprimo con todos los elementos anteriores de la lista. Concluimos que $q$ no estaba en la lista desde un principio.
\end{sol}

\begin{sol}
	Solucion pendiente
\end{sol}

\begin{sol}
	Supongamos lo contrario, es decir existe un primo $q$ tal que $q | 2^{n}+1$ y $q = 8k +7$. Notemos que $\left(\dfrac{2}{q}\right) = 1$ y $\left(\dfrac{-1}{q}\right) =-1$. Pero 
	\begin{align}
	\left(a^{n}\right)^{2} \equiv \left(a^{2}\right)^{n} \equiv 2^{n} \equiv -1 \implies \left(\dfrac{-1}{q}\right) = 1
	\end{align}
	
	Lo cual es una contradicci\'on. 
\end{sol}

\begin{sol}
	Llame $c_{n} = a_{n}b_{n}$. Supongamos por contradicci\'on que $2003|c_{n_{0}}$ (y que es el menor con esta propiedad). Observemos que $c_{n+1} = a_{n}b_{n} = \left(a_{n}b_{n}\right)^{2001} + a_{n}b_{n} + a_{n}^{2002}+b_{n}^{2002}$, entonces, para $k<n_{0}$ obtenemos que $c_{k+1} \equiv c_{k}^{2001}+c_{k} +2 \mod 2003$, ($\varphi(2003) = 2002$). Ponemos $k = n_{0}-1$ en la ecuaci\'on anterior, nos da que $c_{n_{0}-1}^{2001} +c_{n_{0}-1} + 2 \equiv 0 \iff 1+c_{n_{0}-1}^{2} + 2c_{n_{0}-1} \equiv 0 \iff (c_{n_{0}-1}+1)^{2} \equiv 0 \iff c_{n_{0}-1} \equiv -1$. En particular, tenemos que para $n< n_{0}$, $c_{n}c_{n-1} \equiv \left(1+c_{n-1}\right)^{2} \implies \left(\dfrac{c_{n}c_{n-1}}{2003}\right) = 1 \implies \left(\dfrac{c_{n}}{2003}\right) = \left(\dfrac{c_{n-1}}{2003}\right)$. Como $c_{n_{0}-1} = -1 \implies -1 = \left(\dfrac{c_{n_{0}-1}}{2003}\right) = \left(\dfrac{c_{0}}{2003}\right) = \left(\dfrac{4}{2003}\right) = 1$. Lo cual es una contradicci\'on.
\end{sol}