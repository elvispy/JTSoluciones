\section{Ecuaciones funcionales}
A lo largo de estas soluciones, estar\'e usando la notaci\'on est\'andar de equaci\'on funcional. Esto es, usar\'e  $P(a,b)$ para denotar que se har\'an los reemplazos $x = a, y = b$ en la ecuaci\'on original
\begin{sol}
	El reemplazo $P(x/2, x/2)$ nos da que $f(x) = x^2+f(0)$. Podemos verificar que de hecho $f(x) = x^2+a, a \in \mathbb{R}$ satisface la ecuaci\'on funcional.
\end{sol}

\begin{sol}
	El reemplazo $P(x,0)$ nos da que $f(x) = x^3$. La identidad $(x+y)^3 + (x-y)^3 = x^3 +6xy^2+x^3$ confirma que es la \'unica solucion a la ecuaci\'on funcional.
\end{sol}

\begin{sol}
	Hacemos primero el reemplazo $P(\frac{1}{1-x}) \implies f(\frac{1}{1-x}) + f(\frac{1-x}{-x}) = \frac{1}{1-x} $. El reemplazo $P(1-\frac{1}{x}) \implies f(1-\frac{1}{x})+ f(x) = 1-\frac{1}{x}$. Observe que $\frac{1-x}{-x} = 1-\frac{1}{x}$. Sumando la ecuaci\'on original con la \'ultima ecuaci\'n da que $2f(x) + f(\frac{1}{1-x}) + f(1-\frac{1}{x}) = x+1-\frac{1}{x}$. Sumando la ecuaci\'on original con las pimeras dos, da $f(x) +f(\frac{1}{1-x}) + f(1-\frac{1}{x}) = x+\frac{1}{1-x} + 1-\frac{1}{x}$. Restando las \'ultimas dos igualdades obtenidas se tiene que $f(x) = \frac{x}{2} +\frac{x-1}{2x} - \frac{1}{2(1-x)} = \frac{x^3-x+1}{2x(x-1)}$. Una extensiva cuenta que no me atrevo a hacer ahora verifica que en efecto esa es la soluci\'on al problema.
\end{sol}

\begin{sol}
	Reemplazando $P(-x)$ se tiene que $-xf(-x)-2xf(x) =-1$. Restando esta ecuaci\'on de la original, se obtiene que $3x(f(x)+f(-x)) = 0$. Como $x \neq 0$ se tiene que $f(x) = -f(-x)$. Reemplazando la \'ultima identidad en la ecuaci\'on original, sigue que $f(x) = \frac{1}{x}$. Podemos verificar que de hecho, esta es la soluci\'on al problema.
\end{sol}

\begin{sol}
	$P(x,x) \implies f(f(0)) = 0$  $ (\star)$. $P(f(0), 0) \implies f(f(f(0))) = f(f(0)) -f(0) \implies f(0) = 0$. $P(x-y,0) \implies f(f(x-y)) = f(x-y) -f(0) \implies f(x-y) = f(x)-f(y)$ $(\star\star)$. $P(x,0) \implies f(f(x)) = f(x)$.  Ahora probaremos inyectividad: Primero, si $f(z) = 0$, sea $y_{0}$ tal que $f(y_{0}) = z$. Tenemos que $0 = f(z) = f(f(y_{0})) = f(y_{0}) = z$. Luego, $f(z) = 0 \implies z = 0$ $(\star\star\star)$. Supongamos $f(x) = f(y)\implies f(x-y) = 0$ por $(\star\star)$.  Por $(\star\star\star)$, tenemos que $x=y$. Eso concluye la inyectividad. Usando el hecho de que $f(f(x)) = f(x)$ +inyectividad, concluimos que $f(x) = x$. Es f\'acil verificar que la identidad cumple con los requisitos de la ecuci\'on funcional.
\end{sol}

\begin{sol}
	$f(n) + 1 = f(f(f(n))) = f(n+1)\implies f(n) = n+a$, para $n\geq 1$. Pero entonces $f(f(n)) = n+2a = n+1 \implies a = \frac{1}{2}$. Contradicci\'on pues el contradominio de $f$ es $\mathbb{N}$.  
\end{sol}

\begin{sol}
	Sea $A = \mathbb{N} - f(\mathbb{N}), B = f(A)$. Vea que $f(m) = f(n) \implies f(f(m)) = f(f(n)) \implies m=n \implies$ f es inyectiva. Por el ejercicio 1.9 (b) tenemos que $B = f(\mathbb{N}) - f(f(\mathbb{N}))$. Note que $A \cap B = \emptyset$. Vea que $a \subset \{1, 2, \dots 2015 \}$. Y por la inyectividad de $f$, tenemos que $|A| = |B|$ (pues son disjuntos). Finalmente vea que $A\cup B = \mathbb{N} - f(f(\mathbb{N})) = \{1, 2, 3 \cdots 2015 \}$. Pero $2|A| = |A| + |B| = |A \cup B| = 2015$ es una contradici\'on. 
\end{sol}

\begin{sol}
	$P(\frac{x}{2},\frac{x}{2}) \implies f(x)\cdot f(0) = x+1$. $P(0,0) \implies f(0)^2 = 1 \implies f(0) = \pm 1$. Entonces $f(x) = \pm(x+1)$. Verificando las soluciones podemos ver que solo $f(x) = x+1$ satisface. 
\end{sol}