\section{Desigualdad de reacomodo}

\begin{ejer}
	Para $a,b,c \in \mathbb{R}^{+}$, probar que $a^3+b^3+c^3\geq a^2 b+b^2 c+c^2 a$
\end{ejer}
\begin{sol}
	Suponga $a\geq b\geq c $ y aplique reacomodo.
\end{sol}

\begin{ejer}
	Para $a,b,c\in\mathbb{R}^{+}$, probar que
	\begin{equation}
	\frac{a+b+c}{abc} \leq \frac{1}{a^{2}} + \frac{1}{b^{2}}  + \frac{1}{c^{2}} 
	\end{equation}	
\end{ejer}
\begin{sol}
	Suponga $a\geq b\geq c \implies \frac{1}{c} \geq \frac{1}{b} \geq \frac{1}{a}$. Note que
	\begin{align}
	\frac{1}{bc} + \frac{1}{ac} + \frac{1}{ab} = \frac{a+b+c}{abc} \leq \frac{1}{a^2} + \frac{1}{b^2} + \frac{1}{c^2}
	\end{align}
	Debido a la desigualdad de reacomodo.
\end{sol}

\begin{ejer}
	Si $a_{1},a_{2},…,a_{n}\in \mathbb{R}^{+}$ y $s=a_{1}+a_{2}+ \dots +a_{n}$, entonces
	\begin{equation}
	    \frac{a_{1}}{s-a_{1}} + \frac{a_{2}}{s-a_{2}} + \dots + \frac{a_{n}}{s-a_{n}} \geq \frac{n}{n-1}
	\end{equation}
\end{ejer}
\begin{sol}
	Suponga un orden de los factores: $a_{1} \geq a_{2} \geq a_{3} \dots  \geq a_{n}$,  $s-a_{n} \geq s-a_{n-1} \dots  \geq s-a_{1} \implies \frac{1}{s-a_{1}} \geq \frac{1}{s-a_{2}} \dots \geq \frac{1}{s-a_{n}}$. Apliquemos reacomodo $n-1$ veces y sumemos los resultados: (considere siempre $a_{i mod n}$ cuando $i>n$)
	\begin{align}
	\sum_{i=1}^{n} \frac{a_{i}}{s-a_{i}} &\geq  \sum_{i=1}^{n} \frac{a_{i+1}}{s-a_{i}} \\
	\sum_{i=1}^{n} \frac{a_{i}}{s-a_{i}} &\geq  \sum_{i=1}^{n} \frac{a_{i+2}}{s-a_{i}} \\
	\vdots \hspace{1cm}&\hspace{1cm} \vdots \\
	\sum_{i=1}^{n} \frac{a_{i}}{s-a_{i}} &\geq  \sum_{i=1}^{n} \frac{a_{i+n-1}}{s-a_{i}}
	\end{align}
	Que implica $(n-1)\sum_{i=1}^{n} \frac{a_{i}}{s-a_{i}} \geq \sum_{i}^{n} \frac{s-_a{i}}{s-a_{i}} = n \implies \sum_{i=1}^{n} \frac{a_{i}}{s-a_{i}} \geq \frac{n}{n-1}\hspace{0.2cm}\square$
\end{sol}

\begin{ejer}
	Sean $x_{1},x_{2},\dots ,x_{n} \in \mathbb{R}^{+}$. Usando reacomodo, muestre que
	\begin{equation}
	    \sqrt{\frac{x_{1}^{2} + x_{2}^{2} + \dots + x_{n}^{2}}{n}} \geq \frac{x_{1} + x_{2} +\dots + x_{n}}{n}
	\end{equation}
	Con igualdad si y solamente si $x_{1} = x_{2} = \dots = x_{n}$
\end{ejer}

\begin{sol}
	Sea $G = \sqrt[n]{x_{1}x_{2}\dots x_{n}}$. Sean adem\'as $a_{i} = \frac{x_{1}x_{2} \dots x_{i}}{G^{i}}, \forall i$. Por reacomodo tenemos que $\frac{a_{1}}{a_{2}} + \frac{a_{2}}{a_{3}} + \dots \frac{a_{n}}{a_{1}}  \geq \frac{a_{1}}{a_{1}} + \frac{a_{2}}{a_{2}} + \dots \frac{a_{n}}{a_{n}}  = n$, pero $\frac{a_{1}}{a_{2}} + \frac{a_{2}}{a_{3}} + \dots \frac{a_{n}}{a_{1}} = \frac{G}{x_{1}} + \frac{G}{x_{2}} \dots \frac{G}{x_{n}}$. El resultado sigue.   
\end{sol}

\begin{ejer}
	Si $a,b,c \in \mathbb{R}^{+}$ son tales que $a+b+c=1$, probar que $ab+bc+ca\leq 1/3$
\end{ejer}

\begin{sol}
	Vea que $(a+b+c)^2 = 1 \implies ab+bc+ac = \frac{1-a^2-b^2-c^2}{2}$. Entonces $ab+bc+ca \leq \frac{1}{3} \iff \frac{1-a^2-b^2-c^2}{2} \geq \frac{1}{3} \iff \frac{1}{3} \leq a^2+b^2+c^2$. La \'ultima desigualdad sigue de $\frac{a+b+c}{3} \leq \sqrt{\frac{a^2+b^2+c^2}{3}}$
\end{sol}

\begin{ejer}
	Dados números reales $a_1\leq a_2\leq \dots \leq a_n$ y $b_1\leq b_2\leq \dots \leq b_n$, se tiene que
	\begin{equation}
	\frac{a_{1}b_{1} + a_{2}b_{2} + \dots a_{n}b_{n}}{n} \geq \left( \frac{a_{1} + a_{2} + \dots a_{n}}{n} \right)\left( \frac{b_{1} + b_{2} + \dots b_{n}}{n} \right)
	\end{equation}
\end{ejer}

\begin{sol}
	Observe que, por reacomodo, $\sum_{i=1}^{n} a_{i}b_{i} \geq \sum_{i=1}^{n} a_{i+j}b_{i}, \forall i = 1, 2, \dots n$, donde se considera $a_{i} = a_{i+n}, \forall i$. Definamos $S_{1} = \sum_{i=1}^{n} a_{i}, S_{2} = \sum_{i=1}^{n} b_{i}$. Sumando estas $n$ desigualdades, tenemos
	$n(\sum_{i=1}^{n} a_{i}b_{i}) \geq \sum_{i=1}^{n} S_{1}b_{i} = S_{1}\sum_{i=1}^{n} b_{i} = S_{1}S_{2}$. Como queriamos demostrar
\end{sol}

\begin{ejer}
	Sean $a,b,c$ reales positivos tales que $ab+bc+ca=3$. Pruebe que
	\begin{equation}
	\frac{a^{3}}{b+c} + \frac{b^{3}}{a+c} + \frac{c^{3}}{b+a}  \geq \frac{3}{2} 
	\end{equation}
\end{ejer}

\begin{sol}
	Considere un orden: $a \geq b \geq c\implies \frac{1}{b+c} \geq \frac{1}{a+c} \geq \frac{1}{a+b} $. Considere las siguientes desigualdades, todas obtenidas con reacomodo (la ultima usando reacomodo dos veces)
	\begin{align}
	\frac{a^3}{b+c}+\frac{b^3}{a+c} + \frac{c^3}{a+b} & \geq \frac{a^2b}{b+c}+\frac{b^2c}{a+c} + \frac{c^2a}{a+b}\\
	\frac{a^3}{b+c}+\frac{b^3}{a+c} + \frac{c^3}{a+b} & \geq \frac{a^2c}{b+c}+\frac{b^2a}{a+c} + \frac{c^2b}{a+b}\\
	\frac{a^3}{b+c}+\frac{b^3}{a+c} + \frac{c^3}{a+b} & \geq \frac{abc}{b+c}+\frac{abc}{a+c} + \frac{abc}{a+b}
	\end{align}
	
	Sumando todo obtenemos que 
	\begin{align}
	3\bigg(\frac{a^3}{b+c}+\frac{b^3}{a+c} + \frac{c^3}{a+b}\bigg) \geq 3 \bigg(\frac{a}{b+c}+\frac{b}{a+c} + \frac{c}{a+b}\bigg) \geq \frac{9}{2}
	\end{align}
	Donde usamos el hecho de que $ab+bc+ca = 3 $ y la desigualdad de Nesbitt.
\end{sol}

\begin{ejer}
    Sean$ a_1,a_2,a_3,a_4$ reales positivos. Pruebe que
    
    \begin{equation}
    \sum_{1 \leq i < j < k \leq 4} \frac{a_{i}^{3} + a_{j}^{3} + a_{k}^{3}}{a_{i} + a_{j} + a_{k}} \geq a_{1}^{2} + a_{2}^{2} + a_{3}^{2} + a_{4}^{2}
    \end{equation} 
\end{ejer}
\begin{sol}
	\begin{lem}
		\centering {$\frac{a^3+b^3+c^3}{a+b+c} \geq \frac{1}{3} (a^2+b^2+c^2)$}
		
	\end{lem}
	\textbf{\textit{Prueba:}} La afirmaci\'on es equivalente a $3a^3+3b^3+3c^3 \geq (a+b+c)(a^2+b^2+c^2) = a^3+b^3+c^3+ab^2+ac^2+a^2b+ac^2+ca^2+cb^2$ que sigue de usar reacomodo dos veces. $\square$
	
	Tenemos pues 

	\begin{equation}
		\sum_{\substack{1\leq i < j < k \leq 4}} \frac{a_{i}^3+a_{j}^{3}+a_{k}^{3}}{a_{i}+a_{j}+a_{k}} \geq \Sigma_{1\leq i < j < k \leq 4} \frac{a_{i}^2+a_{j}^{2}+a_{k}^{2}}{3} = a_{1}^{2}+a_{2}^{2}+a_{3}^{2}+a_{4}^{2}
	\end{equation} 
\end{sol}

